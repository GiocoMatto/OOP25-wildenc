\documentclass[a4paper,12pt]{report}

\usepackage{alltt, fancyvrb, url}
\usepackage{graphicx}
\usepackage[utf8]{inputenc}
\usepackage{float}
\usepackage{xcolor}
\usepackage{hyperref}

\usepackage[italian]{babel}

\usepackage[italian]{cleveref}

\title{Meta-relazione per\\``Programmazione ad Oggetti''}

\author{
	Andrea Maria Castronovo \\
	Leonardo Mengozzi \\
	Lorenzo Mazzini \\
	Kleo Rama
}

\date{\today}

\begin{document}

\maketitle

\setcounter{page}{2}

\chapter{Analisi}

\section{Descrizione e requisiti}

Wild Encounter è un gioco 2D bullet-heaven\footnote{Categoria di giochi in cui i nemici convergono sul giocatore} ispirato a Vampire Survivors ma ambientato nel mondo Pokémon.
%
Il giocatore si trova sempre al centro dello schermo e deve sopravvivere alle orde di Pokémon selvatici che arrivano da ogni lato. 
%
Man mano che avanza il tempo i nemici diventano sempre più forti e numerosi, aumentando così il grado di sfida. 
%
Affrontando i nemici si guadagna esperienza che permette di salire di livello e di sbloccare nuove MT\footnote{Macchine Techine, ovvero mosse usabili dai Pokémon}.
%
Ogni livello appare una scelta di 3 MT che serviranno per fronteggiare i Pokémon selvatici sempre più potenti e numerosi. 
%
Ogni MT può salire di livello in base a quante volte è scelta al level-up, fino ad arrivare ad un livello massimo.

\subsection{Requisiti funzionali}
\begin{itemize}
	\item Muovere il proprio personaggio all'interno della mappa infinita;
    \item Presenza di Pokémon che cercano di avvicinarsi e danneggiare il giocatore;
    \item Combattere i Pokémon
    \item Aumentare il proprio livello guadagnando esperienza sconfiggento i Pokémon selvatici;
    \item Scegliere tra diversi tipi di MT ad ogni livello;
	\item Possibilità di combattere i Pokémon selvatici tramite MT trovate salendo di livello.
\end{itemize}

\subsection{Requisiti non funzionali}
\begin{itemize}
	\item Progredendo nel gioco saranno presenti a schermo quantità di Pokémon elevate, le prestazioni dovranno però restare accettabili
\end{itemize}


\end{document}
