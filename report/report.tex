\documentclass[a4paper,12pt]{report}

\usepackage{alltt, fancyvrb, url}
\usepackage{graphicx}
\usepackage[utf8]{inputenc}
\usepackage{float}
\usepackage{xcolor}
\usepackage[italian]{hyperref}

\usepackage[italian]{babel}

\usepackage[italian]{cleveref}

\title{Wild Encounter}

\author{
	Andrea Maria Castronovo \\
	Leonardo Mengozzi \\
	Lorenzo Mazzini \\
	Kleo Rama
}

\date{\today}

\begin{document}

\maketitle

\setcounter{page}{2}

\chapter{Analisi}

\section{Descrizione e requisiti}

Wild Encounter è un gioco 2D bullet-heaven\footnote{Categoria di giochi in cui i nemici convergono sul giocatore} ispirato a Vampire Survivors ma ambientato nel mondo Pokémon.
%
Il giocatore si trova sempre al centro dello schermo e deve sopravvivere alle orde di Pokémon selvatici che arrivano da ogni lato. 
%
Man mano che avanza il tempo i nemici diventano sempre più forti e numerosi, aumentando così il grado di sfida. 
%
Affrontando i nemici si guadagna esperienza che permette di salire di livello e di sbloccare nuove MT\footnote{Macchine Techine, ovvero mosse usabili dai Pokémon}.
%
Ogni livello appare una scelta di 3 MT che serviranno per fronteggiare i Pokémon selvatici sempre più potenti e numerosi. 
%
Ogni MT può salire di livello in base a quante volte è scelta al level-up, fino ad arrivare ad un livello massimo.

\subsection{Requisiti funzionali}
\begin{itemize}
	\item Muovere il proprio personaggio all'interno della mappa infinita;
    \item Presenza di Pokémon che cercano di avvicinarsi e danneggiare il giocatore;
    \item Combattere i Pokémon
    \item Aumentare il proprio livello guadagnando esperienza sconfiggento i Pokémon selvatici;
    \item Scegliere tra diversi tipi di MT ad ogni livello;
	\item Possibilità di combattere i Pokémon selvatici tramite MT trovate salendo di livello.
\end{itemize}

\subsection{Requisiti non funzionali}
\begin{itemize}
	\item Progredendo nel gioco saranno presenti a schermo quantità di Pokémon elevate, le prestazioni dovranno però restare accettabili;
	\item Possibilità di tenere conto di quanti Pokémon sono stati visti in ogni partita in un Pokédex.
\end{itemize}

\section{Modello del Dominio}

La \emph{Mappa} di gioco conterrà le varie \emph{Entità} che possono essere animate come 
il \emph{Giocatore} ed i \emph{Nemici} o inanimate come i \emph{Collezionabili}.
%
Durante la partita appariranno casualmente diversi tipi di nemici che arriveranno dai 
bordi dello schermo, inseguendo il giocatore cercando di danneggiarlo.
%
Sia il giocatore che i nemici posseggono delle \emph{Armi}, ciascuna dotata di caratteristiche diverse,
come ad esempio portata, danno e velocità di attacco.
%
Per garantire una corretta progressione del gioco, i nemici, una volta eliminati, rilasceranno 
punti esperienza, monete o bacche per recuperare vita che possono essere collezionati dal giocatore (Collezionabili).
%
Accumulando punti esperienza il giocatore sale di livello, che tornerà a 0 ad ogni partita.
%
Salire di livello permette al giocatore di scegliere delle armi che lo aiuteranno ad
affrontare i nemici sempre più forti e numerosi. Se il giocatore sceglie un'arma che 
possiede già potrà potenziarla aumentandone le statistiche fino ad un certo massimo.
%
La \autoref{mermaid:domain} mostra le relazioni tra gli oggetti descritti

\begin{figure}[H]
	\includegraphics[width=13cm]{mermaid/domain.pdf}
	\caption{Schema UML dell'analisi del dominio, con rappresentate le entità principali ed i rapporti fra loro}\label{mermaid:domain}
\end{figure}

\end{document}
